\documentclass[11pt]{article}

\usepackage[margin=1in]{geometry}
\usepackage{hyperref}
\usepackage{enumitem}
\usepackage{xcolor}
\usepackage{listings}
\usepackage{setspace}

% ---------------- Compact lists ----------------
\setlist[itemize]{noitemsep, topsep=0pt}
\setlist[enumerate]{noitemsep, topsep=0pt}

\setstretch{1.1}

\definecolor{codegray}{gray}{0.95}

\lstset{
  basicstyle=\ttfamily\small,
  backgroundcolor=\color{codegray},
  frame=single,
  breaklines=true,
  showstringspaces=false
}

\title{Quick Start Guide: VS Code, Python, and Jupyter}
\author{Course Computing Setup}
\date{}

\begin{document}
\maketitle

\section*{Goal}
By the end of this guide, you will be able to:
\begin{itemize}
  \item Write and run Python code
  \item Use Jupyter notebooks
  \item Make plots and analyze data
\end{itemize}

All work in this course will be done \textbf{inside Visual Studio Code (VS Code)} using a Python environment named \texttt{geo-env}.

\hrule
\vspace{1em}

\section*{Step 1: Install Visual Studio Code}
Download and install VS Code from:
\begin{center}
\url{https://code.visualstudio.com}
\end{center}

Use the default installation options.

\textbf{Windows users:}  
If the installer asks whether to \emph{Add VS Code to PATH}, check that box.

\section*{Step 2: Install Python (Official Version)}
Install Python from:
\begin{center}
\url{https://www.python.org}
\end{center}

\subsection*{Important for Windows}
During installation:
\begin{itemize}
  \item Check \textbf{``Add Python to PATH''}
  \item Use default installation options
\end{itemize}

\subsection*{Verify Installation (Mac and Windows)}
Open a terminal:
\begin{itemize}
  \item Mac: Applications $\rightarrow$ Utilities $\rightarrow$ Terminal
  \item Windows: Command Prompt or PowerShell
\end{itemize}

Run:
\begin{lstlisting}
python --version
pip --version
\end{lstlisting}

If \texttt{python} does not work on Windows, try:
\begin{lstlisting}
py --version
\end{lstlisting}

\section*{Step 3: Install VS Code Extensions}
Open VS Code, go to the Extensions tab, and install:
\begin{itemize}
  \item \textbf{Python} (Microsoft)
  \item \textbf{Jupyter} (Microsoft)
\end{itemize}

These allow Python scripts and Jupyter notebooks to run inside VS Code.

\section*{Step 4: Create a Course Folder (Finder / File Explorer)}
You do \textbf{not} need the terminal for this step.

\subsection*{Mac}
\begin{enumerate}[leftmargin=*]
  \item Open \textbf{Finder}
  \item Click \textbf{Documents}
  \item Right-click $\rightarrow$ New Folder
  \item Name the folder:
\end{enumerate}

\begin{lstlisting}
earth_history_code
\end{lstlisting}

\begin{enumerate}[leftmargin=*, resume]
  \item Open VS Code
  \item File $\rightarrow$ Open Folder
  \item Select \path{earth_history_code}
\end{enumerate}

\subsection*{Windows}
\begin{enumerate}[leftmargin=*]
  \item Open \textbf{File Explorer}
  \item Click \textbf{Documents}
  \item Right-click $\rightarrow$ New $\rightarrow$ Folder
  \item Name the folder:
\end{enumerate}

\begin{lstlisting}
earth_history_code
\end{lstlisting}

\begin{enumerate}[leftmargin=*, resume]
  \item Open VS Code
  \item File $\rightarrow$ Open Folder
  \item Select \path{earth_history_code}
\end{enumerate}

\textbf{Important:}  
All terminals, environments, scripts, and notebooks must be created inside this folder.

\section*{Step 5: Create the Python Environment (\texttt{geo-env})}
Open the VS Code terminal:
\begin{center}
Terminal $\rightarrow$ New Terminal
\end{center}

\subsection*{Mac}
\begin{lstlisting}
python -m venv geo-env
source geo-env/bin/activate
python -m pip install --upgrade pip
\end{lstlisting}

\subsection*{Windows (PowerShell)}
\begin{lstlisting}
python -m venv geo-env
.\geo-env\Scripts\Activate.ps1
python -m pip install --upgrade pip
\end{lstlisting}

If activation fails, run once:
\begin{lstlisting}
Set-ExecutionPolicy -Scope CurrentUser RemoteSigned
\end{lstlisting}

Then try activating again.

\subsection*{Windows (Command Prompt)}
\begin{lstlisting}
python -m venv geo-env
geo-env\Scripts\activate.bat
python -m pip install --upgrade pip
\end{lstlisting}

\subsection*{Check That It Worked}
\begin{lstlisting}
python --version
\end{lstlisting}

The Python path should include \texttt{geo-env}.

\section*{Step 6: Install Required Packages}
Inside the \path{earth_history_code} folder, create a file named:

\begin{lstlisting}
requirements.txt
\end{lstlisting}

Paste the following:

\begin{lstlisting}
numpy
scipy
pandas
matplotlib

jupyter
ipykernel

xarray
netcdf4
h5py

scikit-learn

scikit-image
pillow

tqdm

ruff
black
pytest
\end{lstlisting}

Install the packages:
\begin{lstlisting}
pip install -r requirements.txt
\end{lstlisting}

\section*{Step 7: Register \texttt{geo-env} as a Jupyter Kernel}
\begin{lstlisting}
python -m ipykernel install --user --name geo-env --display-name "Python (geo-env)"
\end{lstlisting}

\section*{Step 8: Select the Python Interpreter in VS Code}
\begin{enumerate}[leftmargin=*]
  \item Press \textbf{Ctrl+Shift+P} (Windows) or \textbf{Cmd+Shift+P} (Mac)
  \item Run \textbf{Python: Select Interpreter}
  \item Choose the interpreter inside \texttt{geo-env}
\end{enumerate}

\section*{Step 9: Test with a Jupyter Notebook}
In VSCode, go to select File, then New File from the dropdown menu. Then create a new file named:
\begin{lstlisting}
test.ipynb
\end{lstlisting}

In the top-right of the notebook, select:
\begin{center}
\texttt{Python (geo-env)}
\end{center}

Run:
\begin{lstlisting}
import numpy as np
import matplotlib.pyplot as plt

x = np.linspace(0, 10, 200)
plt.plot(x, np.sin(x))
plt.title("geo-env works")
plt.show()
\end{lstlisting}

If you see a plot, your setup is complete.

\section*{Common Issues}

\subsection*{Kernel Not Showing Up}
\begin{lstlisting}
python -m pip install ipykernel
python -m ipykernel install --user --name geo-env --display-name "Python (geo-env)"
\end{lstlisting}
Restart VS Code.

\subsection*{VS Code Using the Wrong Python}
Run \textbf{Python: Select Interpreter} again and choose \texttt{geo-env}.

\subsection*{Permission Errors (Mac)}
\begin{lstlisting}
python -m pip install --upgrade pip setuptools wheel
\end{lstlisting}

\hrule
\vspace{1em}

\section*{You Are Done}
If something breaks later:
\begin{itemize}
  \item Restart VS Code
  \item Activate \texttt{geo-env}
  \item Re-select the interpreter
\end{itemize}

\end{document}
