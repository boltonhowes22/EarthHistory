\documentclass[letterpaper]{tufte-handout}

% --- Packages ---
\usepackage[T1]{fontenc}
\usepackage{lmodern}
\usepackage{amsmath,amssymb}
\usepackage{hyperref}
\usepackage{paralist} % compactenum
\usepackage{booktabs}
\usepackage{ragged2e}

% --- Hyperref settings (optional, but nice) ---
\hypersetup{
  colorlinks=true,
  linkcolor=blue,
  urlcolor=blue,
  citecolor=blue
}

\title{How to Read a Scientific Paper: A Guided Worksheet}
\author{}
\date{}

% --- Small helpers ---
\newcommand{\checkbox}{\(\square\)}
\newcommand{\linefill}{\rule{0.9\linewidth}{0.4pt}}
\newcommand{\shortline}{\rule{0.45\linewidth}{0.4pt}}

\begin{document}
\maketitle

% --- Clickable link at the very top (full width) ---
\begin{fullwidth}
I find this guide helpful when reading scientific papers, take a look:\\
\noindent\href{https://arc.duke.edu/how-to-read-and-understand-a-scientific-paper-a-guide-for-non-scientists/}{How to Read and Understand a Scientific Paper}\\
\vspace{0.25em}
\end{fullwidth}

\vspace{0.5em}

\section*{Paper information}
\begin{compactenum}
  \item Title:
  \item First Author:
  \item \textbf{Why are we reading this paper in this class? (1--2 sentences)}
\end{compactenum}

\section*{Before you begin: reading mindset (2 minutes)}
\begin{compactenum}
  \item \checkbox\ I will \textbf{not} try to understand everything in one pass.
  \item \checkbox\ I will keep a \textbf{running glossary} of unfamiliar terms.
  \item \checkbox\ I will focus on the \textbf{question $\rightarrow$ approach $\rightarrow$ evidence $\rightarrow$ claim}.
  \item \checkbox\ I will look for \textbf{limitations/assumptions}, not just conclusions.
\end{compactenum}

\section*{Step 1: Read the Abstract and Introduction first }
\noindent\textbf{As you read the Abstract and Introduction, answer:}
\vspace{0.25em}

\noindent(a) What problem is the broader field trying to solve? (“the big question”)\\

\noindent(b) What is the gap? What couldn't previous studies do?

\noindent(c) What is the paper's motivation? Why does the gap matter?

\section*{Step 2: Summarize the background in 5 sentences or fewer}
\noindent Write a \textbf{five-sentence max} background summary that sets up the study.\\
\vspace{10em}


\section*{Step 3: Identify the specific research question(s)}
\noindent\textbf{List the specific question(s) the authors claim to answer.}
\begin{compactenum}
  \item Question 1: 
  \item Question 2: 
  \item Question 3: 
\end{compactenum}

\vspace{.5em}
\noindent\textbf{If hypotheses are stated, write them here:}\\
\vspace{5em}

\section*{Step 4: Identify the approach (the game plan)}
\noindent\textbf{In plain language: what do the authors do to answer the question(s)?}\\
\linefill\\[0.6em]

\noindent\textbf{Key data types (check all that apply):}\\
\checkbox\ field observations \quad
\checkbox\ lab measurements \quad
\checkbox\ experiments \quad
\checkbox\ modeling \quad
\checkbox\ compilation/meta-analysis \quad
\checkbox\ other: \shortline

\section*{Step 5: Methods map (draw it)}
\noindent You do \textbf{not} need to replicate the study, but you should be able to explain the workflow.
\begin{compactenum}
  \item What are the samples / study system?
  \item What are the key variables measured? 
  \item What is the comparison / test?
  \item What are the key assumptions in the methods?
\end{compactenum}

\noindent\textbf{Draw a simple diagram of the workflow here (boxes + arrows):}\\
\vspace{0.3em}
\begin{fullwidth}
\noindent\framebox[\linewidth][l]{\parbox{\linewidth}{\vspace{2.4in}}}
\end{fullwidth}

\section*{Step 6: Results inventory (figure-by-figure)}
\noindent\textbf{Do not interpret yet. Just record what the results are.}\\
\vspace{0.5em}

\begin{fullwidth}
\begin{tabular}{@{}p{0.10\linewidth}p{0.42\linewidth}p{0.42\linewidth}@{}}
\toprule
\textbf{Fig/Table} & \textbf{What was measured / shown?} & \textbf{What is the key pattern / number?} \\
\midrule
1 & & \\
\addlinespace[1.2em]
2 & & \\
\addlinespace[1.2em]
3 & & \\
\addlinespace[1.2em]
4 & & \\
\addlinespace[1.2em]
5 & & \\
\addlinespace[1.2em]
\bottomrule
\end{tabular}
\end{fullwidth}

\section*{Step 7: Do the results answer the specific question(s)? (your interpretation)}
\noindent\textbf{For each specific question, answer: “yes/no/partly” and explain why.}
\begin{compactenum}
  \item Question 1: 
  \item Question 2: 
  \item Question 3: 
\end{compactenum}

\begin{fullwidth}
\noindent\textbf{Your best one-paragraph interpretation (before reading Discussion):}\\
\vspace{0.3em}

\noindent\framebox[\linewidth][l]{\parbox{\linewidth}{\vspace{3in}}}    
\end{fullwidth}

\section*{Step 8: Read the Discussion/Conclusion (authors’ interpretation)}
\noindent\textbf{(a) What are the authors' main claims? (list 2--4)}\\
\begin{compactenum}
  \item 
  \item 
  \item 
\end{compactenum}

\noindent(b) What evidence supports each claim? (cite figure/table numbers)\\


\noindent(c) What limitations do the authors acknowledge?\\

\noindent(d) What limitations/assumptions do \emph{you} think matter most?\\


\noindent(e) What do they propose as the next step?\\

\section*{Step 9: Now read the Abstract (last)}
\noindent\textbf{Does the abstract match what the paper actually shows?}\\
\checkbox\ yes \quad \checkbox\ mostly \quad \checkbox\ not really

\vspace{0.4em}
\noindent\textbf{One thing the abstract emphasized that you think is overstated or unclear:}\\

\noindent\textbf{One important nuance the abstract underplays or omits:}\\


\section*{Final takeaway (the 30-second explanation)}
\noindent Imagine you’re explaining this paper to a classmate who missed the reading. Fill in:

\vspace{0.4em}
\noindent\textbf{This paper asks:} \linefill\\[0.6em]
\noindent\textbf{They test it by:} \linefill\\[0.6em]
\noindent\textbf{They find:} \linefill\\[0.6em]
\noindent\textbf{So they argue:} \linefill\\[0.6em]
\noindent\textbf{The biggest caveat is:} \linefill\\[0.6em]

% --- Make tufte-handout/bibentry + latexmk/BibTeX happy (without printing a bibliography) ---
\bibliographystyle{plainnat}
\nobibliography{../bibliography} % <- change this path to wherever your .bib lives




\end{document}
